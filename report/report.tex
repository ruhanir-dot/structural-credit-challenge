\documentclass[12pt]{article}
\usepackage[utf8]{inputenc}
\usepackage[margin=1in]{geometry}
\usepackage{amsmath}
\usepackage{amssymb}
\usepackage{bm}
\usepackage{graphicx}
\usepackage{booktabs}
\usepackage{float}
\usepackage{hyperref}
\usepackage{caption}
\usepackage{subcaption}
\title{Technical Report: Merton Structural Credit Model \\
\large{with Exponential Smoothing Improvement}}
\author{Ruhani Rekhi}
\date{December 25, 2025}
\begin{document}
\maketitle
\section{Introduction}
This report presents an implementation and improvement of the Merton (1974) 
structural credit model. After implementing and diagnosing the baseline model on API-fetched 2020 data for five U.S. firms (Apple, JPMorgan Chase, Tesla, ExxonMobil, and Ford), 
we identified default probability (PD) instability as a primary weakness: all firms exhibit coefficient of variation (CV) $> 1.0$, making the model unsuitable
for daily credit risk monitoring.

To approach this issue we implement exponential smoothing as a minimal improvement that reduces PD volatility by 15--20\% and daily changes by 25--50\% 
while preserving the model's ability to properly capture credit deterioration during crisis periods.

\section{Model Formulation}
The Merton model views a firm's equity as a European call option on its assets, with the strike price being the face value of its debt.
The key equations and assumptions of the baseline Merton model are outlined below.

\subsection{Baseline Merton Model}

\subsubsection{Key Assumptions}
\begin{enumerate}
    \item Firm asset value $V_t$ follows geometric Brownian motion: 
    \begin{equation}
        dV_t = \mu V_t \, dt + \sigma_V V_t \, dW_t
    \end{equation}
    \item Firm has zero-coupon debt with face value $D$ maturing at time $T$
    \item Default occurs only at maturity $T$ if $V_T < D$
    \item Equity is a European call option on firm assets: $E = \max(V_T - D, 0)$
    \item Markets are frictionless (no transaction costs), trading is continuous, no arbitrage(no riskless profit opportunities)
\end{enumerate}

\subsubsection{Key Equations/ Mathematical Formulation}
\paragraph{Equity Value} Under the Merton model, equity value is given by the Black-Scholes call option formula:
\begin{equation}
    E = V \cdot \Phi(d_1) - D \cdot e^{-r(T-t)} \cdot \Phi(d_2)
\end{equation}
where
\begin{align}
    d_1 &= \frac{\ln(V/D) + (r + \sigma_V^2/2)(T-t)}{\sigma_V\sqrt{T-t}} \\
    d_2 &= d_1 - \sigma_V\sqrt{T-t}
\end{align}
and $\Phi(\cdot)$ denotes the standard normal cumulative distribution function.

\paragraph{Equity Volatility} The relationship between equity volatility $\sigma_E$ and asset volatility $\sigma_V$ is:
\begin{equation}
    \sigma_E \cdot E = \Phi(d_1) \cdot \sigma_V \cdot V
\end{equation}

where $\Phi(d_1)$ is the option delta, representing the sensitivity of equity value to changes in asset value.

\paragraph{Default Probability} The risk-neutral probability of default at maturity is:
\begin{equation}
    \text{PD} = \Phi(-d_2) = \Phi\left(-\frac{\ln(V/D) + (r - \sigma_V^2/2)T}{\sigma_V\sqrt{T}}\right)
\end{equation}

\paragraph{Distance-to-Default} The distance-to-default metric is:
\begin{equation}
    \text{DD} = \frac{\ln(V/D) + (r - \sigma_V^2/2)T}{\sigma_V\sqrt{T}}
\end{equation}

\subsubsection{Calibration Approach}
The Merton model calibration involves solving the following system of equations for the unobservable asset value $V$ and asset volatility $\sigma_V$,
given the observable equity value $E$, equity volatility $\sigma_E$, debt face value $D$, time to maturity $T-t$, and risk-free rate $r$:
\begin{align}
    E &= \text{BlackScholes}(V, D, T-t, r, \sigma_V) \\
    \sigma_E \cdot E &= \Phi(d_1) \cdot \sigma_V \cdot V
\end{align}

\subsection{Improved Model}

\subsubsection{Modification}
We apply post-processing exponential smoothing to default probabilities. For each firm's time series of raw PDs $\{\text{PD}_t^{\text{raw}}\}$ we applied the following smoothing formula:
\begin{equation}
    \text{PD}_t^{\text{smooth}} = \alpha \cdot \text{PD}_t^{\text{raw}} + (1 - \alpha) \cdot \text{PD}_{t-1}^{\text{smooth}}
\end{equation}
where $\alpha = 0.1$ is the smoothing parameter.

\subsubsection{Justification}

This improvement is justified on two grounds. Theoretically speaking, the baseline model assumes you can trade infinitely with no transaction costs. Whereas in reality, trading is discrete 
and every trade has costs leading to noise in observed prices that isn't properly captured in the model. Additionally, daily equity volatility contains high-frequency noise (changes in company value, as well as market microstructure noise) which is
further amplified through the nonlinear Black-Scholes model into asset volatility thus into PD estimates. 
Practically speaking, PD estimates play a key role in credit risk management for portfolio monitoring, capital alllocation, and risk limits. When daily PD estimates 
oscillate wildly (for example 5-20 percent points as seen in the baseline model), they become operationally unusable. 
This is why smoothing techniques are applied in practice to reduce noise while preserving signal.

\subsubsection{Mathematical Formulation of Improvement}

The exponential weighted moving average (EWMA), the smoothing formula we are using, has the following properties.
\begin{itemize}
    \item Parameter $\alpha$ controls reactivity: lower $\alpha$ yields more smoothing
    \item No look-ahead bias (uses only past observations)
    \item Preserves long-term trends while filtering short-term noise
    \item Weights decay exponentially: observation $k$ periods ago has weight $\alpha(1-\alpha)^k$
\end{itemize}

Diving into the math, property 1 of the EWMA smoothing formula which is equation (10) can be demonstrated by unrolling the recursion:
\begin{equation}
    \text{PD}_t^{\text{smooth}} = \alpha \sum_{k=0}^{\infty} (1-\alpha)^k \cdot \text{PD}_{t-k}^{\text{raw}}
\end{equation}
This shows how $\alpha$ controls the weight decay rate, thus the reactivity. For example, with $\alpha = 0.9$ the decay factor
$(1-\alpha) = 0.1$ which means today would be 90\%, yesterday 9\%, two days ago 0.9\%, etc. Showing that the higher alpha 
has high reactivity to recent changes since recent data dominates. Conversely, with $\alpha = 0.1$ the decay factor $(1-\alpha) = 0.9$ which means today would be 10\%, 
yesterday 9\%, two days ago 8.1\%, etc. This shows that lower alpha has low reactivity since weights are more evenly spread out over time.

Property 2 is evident since the formula only uses $\text{PD}_t^{\text{raw}}$ and $\text{PD}_{t-1}^{\text{smooth}}$ which is computed from past raw PDs.
Additionally the code implementation processes data chronologically to avoid look-ahead bias.

Property 3 is evident since the EWMA formula averages over past raw PDs, thus filtering out high-frequency noise while preserving long-term trends.

Property 4 is shown in equation (11) where the weight for observation $k$ periods ago is $\alpha(1-\alpha)^k$, demonstrating the exponential decay.


\subsubsection{Calibration Changes}

Smoothing is applied post-calibration. The algorithm is:
\begin{enumerate}
    \item Calibrate $(V, \sigma_V)$ using baseline approach
    \item Compute $\text{PD}_t^{\text{raw}}$ from equation (6)
    \item Apply EWMA smoothing to obtain $\text{PD}_t^{\text{smooth}}$
\end{enumerate}

Note that asset values $V$, asset volatilities $\sigma_V$, and distance-to-default remain unchanged from the baseline model.


\end{document}
