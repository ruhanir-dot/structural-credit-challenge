\documentclass[12pt]{article}
\usepackage[utf8]{inputenc}
\usepackage[margin=1in]{geometry}
\usepackage{amsmath}
\usepackage{amssymb}
\usepackage{bm}
\usepackage{graphicx}
\usepackage{booktabs}
\usepackage{float}
\usepackage{hyperref}
\usepackage{caption}
\usepackage{subcaption}
\title{Technical Report: Merton Structural Credit Model \\
\large{with Exponential Smoothing Improvement}}
\author{Ruhani Rekhi}
\date{December 24, 2025}
\begin{document}
\maketitle
\section{Introduction}
This report presents an implementation and improvement of the Merton (1974) 
structural credit model for estimating default probabilities. The baseline model treats equity 
as a call option on firm assets and calibrates unobservable asset values and volatilities from market-observable equity prices (stock prices) and 
volatilities. After implementing and diagnosing the baseline model on 2020 data for five U.S. firms (Apple, JPMorgan Chase, Tesla, ExxonMobil, and Ford), 
we identified \textbf{default probability (PD) instability} as the primary weakness: all firms exhibit coefficient of variation (CV) $> 1.0$, making the model unsuitable
 for daily credit risk monitoring.

To approach this issue we implement \textbf{exponential smoothing} ($\alpha = 0.1$) as a minimal improvement that reduces PD volatility by 15--20\% and daily changes by 25--50\% 
while preserving the model's ability to properly capture credit deterioration during crisis periods.
\end{document}

\section{Model Formulation}

\subsection{Baseline Merton Model}
The Merton Model is a 